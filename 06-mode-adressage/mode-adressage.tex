% === Cours de Java
% === Version couleur pour la présentation

\documentclass[14pt]{beamer}
\usetheme{Warsaw}

\usepackage{mic}

\title{Microprocesseurs (MIC)}
\subtitle{Chap. 6: Les modes d'adressage}
\date{} 

% =====================   DOCUMENT ===================++
\begin{document}

% ---------------------   FRAME   --------------------
\begin{frame}
\titlepage
\end{frame}

% ---------------------   FRAME   --------------------
\begin{frame}[fragile]{Adresser les données}
% ----------------------------------------------------
De nombreuses instructions manipulent des données
	\begin{itemize}
	\item transferts mémoire-regitres : \asm{MOV AX, 3}
	\item calculs : \asm{ADD EAX, EBX}
	\item tests : \asm{BT AX, 2}
	\item \dots
	\end{itemize}

\medskip
Il faut pouvoir indiquer où se trouve / où mettre chaque donnée

\medskip
C'est ce qu'on appelle les \emph{modes d'adressage}
\end{frame}

% ---------------------   FRAME   --------------------
\begin{frame}[fragile]{Modes de base}
% ----------------------------------------------------
\begin{description}
  \item[\emph{Immédiat}] 
	la donnée est directement dans l'instruction
	\\ex: \asm{MOV AX,} \fbox{\asm{42}}
  \item[\emph{Registre}] la donnée est (doit être placée) dans un registre
	\\ex: \asm{MOV} \fbox{\asm{AX}}\asm{,} \fbox{\asm{BX}}
  \item[\emph{Direct}] la donnée est (doit être placée) à l'adresse donnée dans l'instruction
	\\ex: \asm{MOV AX,} \fbox{\asm{[0xB8A0]}}
	\\ex: \asm{MOV} \fbox{\asm{[0xB8A2]}}\asm{, AX}
\end{description}
\end{frame}

% ---------------------   FRAME   --------------------
\begin{frame}[fragile]{Modes de base -- Piège}
% ----------------------------------------------------
Attention à ne pas confondre \emph{une adresse} et \emph{ce qu'elle contient}
\begin{Asm}
	MOV EAX, 0xB8A0      ; immédiat. EAX reçoit la valeur 0xB8A0.
	MOV EAX, [0xB8A0]    ; direct. EAX reçoit la valeur (4 bytes) à l'adresse 0xB8A0.
	MOV AX, [0xB8A0]     ; direct. AX reçoit la valeur (2 bytes) à l'adresse 0xB8A0.
	MOV AL, [0xB8A0]     ; direct. AL reçoit la valeur (1 byte) à l'adresse 0xB8A0.
\end{Asm}
\bigskip
D'ailleurs
\begin{Asm}
	MOV AL, 0xB8A0      ; Ne compile pas ! AL trop petit pour y mettre la valeur
	MOV 0xB8A0, AL      ; Ne compile pas ! Pas d'immédiat à gauche.
\end{Asm}
\end{frame}

% ---------------------   FRAME   --------------------
\begin{frame}[fragile]{Modes de base -- Autre piège}
% ----------------------------------------------------
Un \emph{label} est un nom symbolique pour une \emph{adresse}
\begin{Asm}
	MOV EAX, brol     ; immédiat. On met dans EAX l'adresse brol.
	MOV EAX, [brol]   ; direct. EAX reçoit la valeur (4 bytes) à l'adresse brol.
	MOV AX, [brol]    ; direct. AX reçoit la valeur (2 bytes) à l'adresse brol.
	MOV AL, [brol]    ; direct. AL reçoit la valeur (1 byte) à l'adresse brol.
\end{Asm}
\bigskip
D'ailleurs
\begin{Asm}
	MOV [brol], AX    ; on met le contenu de AX à l'adresse brol
	MOV brol, AX      ; Ne compile pas ! On ne peut pas modifier un label.
\end{Asm}
\end{frame}

% ---------------------   FRAME   --------------------
\begin{frame}[fragile]{Modes de base -- Utilisation}
% ----------------------------------------------------
Ces trois modes permettent de traduire les instructions simples
des langages de haut niveau.

\medskip
Par exemple
\begin{Code}
	b <- a + 2
\end{Code}
pourrait se traduire%
\footnote{
	\scriptsize{Valable pour des variables globales.
	En pratique, a et b sont probablement des variables locales;
	elles seront dès lors stockées dans une pile, ce qui modifie
	un peu les instructions.}
}
\begin{Asm}
	MOV AX, [a]   ; registre, direct
	ADD AX, 2     ; registre, immédiat
	MOV [b], AX   ; direct, registre
\end{Asm}
\end{frame}

% ---------------------   FRAME   --------------------
\begin{frame}[fragile]{RISC vs CISC}
% ----------------------------------------------------

\begin{footnotesize}
\begin{tabular}{p{3cm}p{3cm}p{3cm}}
	& \emph{RISC} & \emph{CISC} \\\hline
Signification & Reduced Instruction Set Computer & Complex Instruction Set Computer \\\hline
Modes d'addressage & Peu & Beaucoup \\\hline
%\# d'insctructions & Plus & Moins \\\hline
Processeur & Moins complexe & Plus complexe \\\hline
\end{tabular}
\end{footnotesize}

\bigskip
Les architectures \emph{CISC} permettent de coder plus facilement
des instructions de haut niveau comme
\begin{Code}
	maStructure.unChamp <- 1
	tab[3] <- 2
	tab[3].unChamp <- 3
\end{Code}
\end{frame}

% ---------------------   FRAME   --------------------
\begin{frame}[fragile]{Adressage indirect (registre)}
% ----------------------------------------------------

La donnée est à une adresse donnée par un registre

\bigskip
Exemple :
\begin{Asm}
	MOV EBX, 0xB8A0      ; EBX contient l'adresse donnée
	MOV EAX, §\textcolor{brown}{\textbf{[EBX]}}§      ; on met dans EAX la donnée se trouvant à l'adresse 0xB8A0
\end{Asm}

\bigskip
Utile pour traduire les instructions manipulant les \emph{pointeurs / références}

\bigskip
Exemple : Pour traduire \code{objet1 <- objet2} (copie des références), 
on pourrait avoir
\begin{Asm}
	MOV EAX, [objet2]      ; l'objet référencé par objet2
	MOV [objet1], EAX      ; et si on ne met pas les crochets ?
\end{Asm}
\end{frame}

% ---------------------   FRAME   --------------------
\begin{frame}[fragile]{Adressage indirect avec déplacement}
% ----------------------------------------------------

Adresse obtenue en ajoutant un déplacement à une adresse dans un registre

\bigskip
Exemple :
\begin{Asm}
	MOV AX, §\textcolor{brown}{\textbf{[EBX + 4]}}§    ; AX reçoit la donnée se trouvant à l'adresse
	                      ; donnée par (le contenu de) EBX + 4
\end{Asm}

\bigskip
Utile pour traduire les instructions manipulant les \emph{structures}

\bigskip
Exemple : Pour traduire \code{maStructure.unChamp <- 1}, on pourrait avoir
\begin{Asm}
	MOV EAX, maStructure   ; pas de crochet ici 
	MOV [EAX + 8], 1        ; Le '8' dépend des champs précédents dans la structure 
\end{Asm}
\end{frame}

% ---------------------   FRAME   --------------------
\begin{frame}[fragile]{Adressage indirect indexé}
% ----------------------------------------------------

Le déplacement est lui aussi dans un registre. On y applique un facteur multiplicatif.

\bigskip
Exemple :
\begin{Asm}
	MOV AX, §\textcolor{brown}{\textbf{[EBX + 4 * ECX]}}§    ; AX reçoit la donnée se trouvant à l'adresse
	                     ; donnée par (le contenu de) EBX + 4 * (le contenu de) ECX 
\end{Asm}

\bigskip
Utile pour traduire les instructions manipulant les \emph{tableaux}

\bigskip
Exemple : Pour traduire \code{tab[3] <- 1}, on pourrait avoir
\begin{Asm}
	MOV EAX, tab               
	MOV EBX,  3 
	MOV [EAX + 4 * EBX], 1         ; Le '4' est la taille d'une case.
	; on comprend mieux que les tableaux commencent à 0 dans de nombreux langages
\end{Asm}
\end{frame}

% ---------------------   FRAME   --------------------
\begin{frame}[fragile]{Remarques}
% ----------------------------------------------------

Nous n'avons pas tout dit :
\begin{itemize}
\item Les modes portent parfois d'autres noms
\item Il existe encore d'autres modes d'adressages
\item Chaque processeur dispose d'un sous-ensemble des modes possibles
\item Chaque assembleur dispose de sa propre syntaxe pour ces modes d'adressages
\end{itemize}
\end{frame}

% ------------------------------------------------------
\end{document}
